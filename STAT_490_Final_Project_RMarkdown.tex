\documentclass[]{article}
\usepackage{lmodern}
\usepackage{amssymb,amsmath}
\usepackage{ifxetex,ifluatex}
\usepackage{fixltx2e} % provides \textsubscript
\ifnum 0\ifxetex 1\fi\ifluatex 1\fi=0 % if pdftex
  \usepackage[T1]{fontenc}
  \usepackage[utf8]{inputenc}
\else % if luatex or xelatex
  \ifxetex
    \usepackage{mathspec}
  \else
    \usepackage{fontspec}
  \fi
  \defaultfontfeatures{Ligatures=TeX,Scale=MatchLowercase}
\fi
% use upquote if available, for straight quotes in verbatim environments
\IfFileExists{upquote.sty}{\usepackage{upquote}}{}
% use microtype if available
\IfFileExists{microtype.sty}{%
\usepackage{microtype}
\UseMicrotypeSet[protrusion]{basicmath} % disable protrusion for tt fonts
}{}
\usepackage[margin=1in]{geometry}
\usepackage{hyperref}
\hypersetup{unicode=true,
            pdftitle={Stat 490 Final Project: Predicting House Prices},
            pdfauthor={Your Name},
            pdfborder={0 0 0},
            breaklinks=true}
\urlstyle{same}  % don't use monospace font for urls
\usepackage{color}
\usepackage{fancyvrb}
\newcommand{\VerbBar}{|}
\newcommand{\VERB}{\Verb[commandchars=\\\{\}]}
\DefineVerbatimEnvironment{Highlighting}{Verbatim}{commandchars=\\\{\}}
% Add ',fontsize=\small' for more characters per line
\usepackage{framed}
\definecolor{shadecolor}{RGB}{248,248,248}
\newenvironment{Shaded}{\begin{snugshade}}{\end{snugshade}}
\newcommand{\KeywordTok}[1]{\textcolor[rgb]{0.13,0.29,0.53}{\textbf{#1}}}
\newcommand{\DataTypeTok}[1]{\textcolor[rgb]{0.13,0.29,0.53}{#1}}
\newcommand{\DecValTok}[1]{\textcolor[rgb]{0.00,0.00,0.81}{#1}}
\newcommand{\BaseNTok}[1]{\textcolor[rgb]{0.00,0.00,0.81}{#1}}
\newcommand{\FloatTok}[1]{\textcolor[rgb]{0.00,0.00,0.81}{#1}}
\newcommand{\ConstantTok}[1]{\textcolor[rgb]{0.00,0.00,0.00}{#1}}
\newcommand{\CharTok}[1]{\textcolor[rgb]{0.31,0.60,0.02}{#1}}
\newcommand{\SpecialCharTok}[1]{\textcolor[rgb]{0.00,0.00,0.00}{#1}}
\newcommand{\StringTok}[1]{\textcolor[rgb]{0.31,0.60,0.02}{#1}}
\newcommand{\VerbatimStringTok}[1]{\textcolor[rgb]{0.31,0.60,0.02}{#1}}
\newcommand{\SpecialStringTok}[1]{\textcolor[rgb]{0.31,0.60,0.02}{#1}}
\newcommand{\ImportTok}[1]{#1}
\newcommand{\CommentTok}[1]{\textcolor[rgb]{0.56,0.35,0.01}{\textit{#1}}}
\newcommand{\DocumentationTok}[1]{\textcolor[rgb]{0.56,0.35,0.01}{\textbf{\textit{#1}}}}
\newcommand{\AnnotationTok}[1]{\textcolor[rgb]{0.56,0.35,0.01}{\textbf{\textit{#1}}}}
\newcommand{\CommentVarTok}[1]{\textcolor[rgb]{0.56,0.35,0.01}{\textbf{\textit{#1}}}}
\newcommand{\OtherTok}[1]{\textcolor[rgb]{0.56,0.35,0.01}{#1}}
\newcommand{\FunctionTok}[1]{\textcolor[rgb]{0.00,0.00,0.00}{#1}}
\newcommand{\VariableTok}[1]{\textcolor[rgb]{0.00,0.00,0.00}{#1}}
\newcommand{\ControlFlowTok}[1]{\textcolor[rgb]{0.13,0.29,0.53}{\textbf{#1}}}
\newcommand{\OperatorTok}[1]{\textcolor[rgb]{0.81,0.36,0.00}{\textbf{#1}}}
\newcommand{\BuiltInTok}[1]{#1}
\newcommand{\ExtensionTok}[1]{#1}
\newcommand{\PreprocessorTok}[1]{\textcolor[rgb]{0.56,0.35,0.01}{\textit{#1}}}
\newcommand{\AttributeTok}[1]{\textcolor[rgb]{0.77,0.63,0.00}{#1}}
\newcommand{\RegionMarkerTok}[1]{#1}
\newcommand{\InformationTok}[1]{\textcolor[rgb]{0.56,0.35,0.01}{\textbf{\textit{#1}}}}
\newcommand{\WarningTok}[1]{\textcolor[rgb]{0.56,0.35,0.01}{\textbf{\textit{#1}}}}
\newcommand{\AlertTok}[1]{\textcolor[rgb]{0.94,0.16,0.16}{#1}}
\newcommand{\ErrorTok}[1]{\textcolor[rgb]{0.64,0.00,0.00}{\textbf{#1}}}
\newcommand{\NormalTok}[1]{#1}
\usepackage{graphicx,grffile}
\makeatletter
\def\maxwidth{\ifdim\Gin@nat@width>\linewidth\linewidth\else\Gin@nat@width\fi}
\def\maxheight{\ifdim\Gin@nat@height>\textheight\textheight\else\Gin@nat@height\fi}
\makeatother
% Scale images if necessary, so that they will not overflow the page
% margins by default, and it is still possible to overwrite the defaults
% using explicit options in \includegraphics[width, height, ...]{}
\setkeys{Gin}{width=\maxwidth,height=\maxheight,keepaspectratio}
\IfFileExists{parskip.sty}{%
\usepackage{parskip}
}{% else
\setlength{\parindent}{0pt}
\setlength{\parskip}{6pt plus 2pt minus 1pt}
}
\setlength{\emergencystretch}{3em}  % prevent overfull lines
\providecommand{\tightlist}{%
  \setlength{\itemsep}{0pt}\setlength{\parskip}{0pt}}
\setcounter{secnumdepth}{0}
% Redefines (sub)paragraphs to behave more like sections
\ifx\paragraph\undefined\else
\let\oldparagraph\paragraph
\renewcommand{\paragraph}[1]{\oldparagraph{#1}\mbox{}}
\fi
\ifx\subparagraph\undefined\else
\let\oldsubparagraph\subparagraph
\renewcommand{\subparagraph}[1]{\oldsubparagraph{#1}\mbox{}}
\fi

%%% Use protect on footnotes to avoid problems with footnotes in titles
\let\rmarkdownfootnote\footnote%
\def\footnote{\protect\rmarkdownfootnote}

%%% Change title format to be more compact
\usepackage{titling}

% Create subtitle command for use in maketitle
\providecommand{\subtitle}[1]{
  \posttitle{
    \begin{center}\large#1\end{center}
    }
}

\setlength{\droptitle}{-2em}

  \title{Stat 490 Final Project: Predicting House Prices}
    \pretitle{\vspace{\droptitle}\centering\huge}
  \posttitle{\par}
    \author{Your Name}
    \preauthor{\centering\large\emph}
  \postauthor{\par}
      \predate{\centering\large\emph}
  \postdate{\par}
    \date{Presentation and Paper: Aug 2, 2018}


\begin{document}
\maketitle

\subsubsection{Important Dates}\label{important-dates}

\begin{itemize}
\item
  Tuesday, Aug 31: Email me an annotated R (or R Markdown) output on
  which your report will be based along with a few paragraphs describing
  your results. If you have any issues about what you should do, write
  them down for me and I will discuss them with you.
\item
  Thursday, Aug 2: Project Presentation
\end{itemize}

\subsubsection{Project Description}\label{project-description}

The final report for the project should be a 3-5 page paper that
describes the questions of interest, how you used the method with
details on the steps you used in your analysis, your findings about your
question of interest and the limitations of your study. Specifically,
your report should contain the following:

\begin{enumerate}
\def\labelenumi{\arabic{enumi}.}
\item
  Abstract: A one paragraph summary of what you set out to learn, and
  what you ended up finding. It should summarize the entire report.
\item
  Introduction: A brief description of the data set, variables, etc.
  Desribe also the main prediction problem and which variables do you
  initially suspect will be associated with the response.
\item
  Analysis: Describe the necessary steps taken to implement procedure.
  Be catalog what you have seen in your exploratory data analysis.
\item
  Results: Provide inferences about the questions of interest and
  discussion.
\item
  Limitations of study and conclusion: Describe any limitations of your
  study and how they might be overcome and provide brief conclusions
  about the results of your study.
\end{enumerate}

\subsubsection{Project Presentation}\label{project-presentation}

Prepare a 10-15 minutes worth of presentation slides which contains
significant portions of your project.

\subsubsection{Seatle Housing Data Set}\label{seatle-housing-data-set}

\begin{enumerate}
\def\labelenumi{\alph{enumi})}
\tightlist
\item
  This dataset contains house sale prices for King County, which
  includes Seattle. It includes homes sold between May 2014 and May
  2015. The main variable of interest is the quntitative variable
  \texttt{price} at which the house was sold. Use the techniques you
  have learned in the class to construct several models predicting
  housing \texttt{price} using the predictors found in the data set. The
  original data \texttt{kc\_house\_data.csv} contains 21,613 rows and 21
  columns. The variables in the data set are the following:
\end{enumerate}

\begin{itemize}
\tightlist
\item
  id - Unique ID for each home sold
\item
  date - Date of the home sale
\item
  price - Price of each home sold
\item
  bedrooms - Number of bedrooms
\item
  bathrooms - Number of bathrooms, where .5 accounts for a room with a
  toilet but no shower
\item
  sqft\_living - Square footage of the apartments interior living space
\item
  sqft\_lot - Square footage of the land space
\item
  floors - Number of floors
\item
  waterfront - A dummy variable for whether the apartment was
  overlooking the waterfront or not 1’s represent a waterfront
  property, 0’s represent a non-waterfront property
\item
  view - An index from 0 to 4 of how good the view of the property was,
  0 - lowest, 4 - highest
\item
  condition - An index from 1 to 5 on the condition of the apartment, 1
  - lowest, 4 - highest
\item
  grade - An index from 1 to 13, where 1-3 falls short of building
  construction and design, 7 has an average level of construction and
  design, and 11-13 have a high quality level of construction and
  design.
\item
  sqft\_above - The square footage of the interior housing space that is
  above ground level
\item
  sqft\_basement - The square footage of the interior housing space that
  is below ground level
\item
  yr\_built - The year the house was initially built
\item
  yr\_renovated - The year of the house’s last renovation
\item
  zipcode - What zipcode area the house is in
\item
  lat - Lattitude
\item
  long - Longitude
\item
  sqft\_living15 - The square footage of interior housing living space
  for the nearest 15 neighbors
\item
  sqft\_lot15 - The square footage of the land lots of the nearest 15
  neighbors
\end{itemize}

\begin{enumerate}
\def\labelenumi{\alph{enumi})}
\setcounter{enumi}{1}
\item
  Be careful with the two natural categorical variables in the data:
  \texttt{id} (numeric when you first load the data) and \texttt{date}.
  You may delete these two variables but note that these may be useful
  depending on the progress of your modelling (for example when you want
  to identify influential observations or when you want to see date of
  purchase or compute the average sales for a given period of time).
\item
  Explore your data by computing summary measures and visualize the
  relationships. You may also create or combine new variables that you
  think is necessary. For example, you can create a variable called
  \texttt{bed\_bath\_ratio} which is the quotient of variables
  \texttt{bedrooms} and \texttt{bathrooms}. Check also the range of
  values in your variables. Often, when a certain predictor has wide
  range you might need to do transformation on this predictor (such as
  \texttt{log} or \texttt{sqrt}) to avoid high leverage (large x values)
  points in your model.
\item
  After you have explored your data and pre-processed your predictor
  variables. Fit several models (linear regression, knn, regression
  tree, and random forest) and perform variable selection (selection
  methods, regularization) using the training data. When you fit the
  model use cross-validation technique especially when using the
  \texttt{train} function in the package \texttt{caret}. Check the
  performance (RMSE) of your models on the test data.
\end{enumerate}

\subsubsection{Loading the data, Separate Training data and Validation
Test
data}\label{loading-the-data-separate-training-data-and-validation-test-data}

\begin{Shaded}
\begin{Highlighting}[]
\CommentTok{# download the data set from Blackboard}
\NormalTok{house_data <-}\StringTok{ }\KeywordTok{read.csv}\NormalTok{(}\StringTok{"kc_house_data.csv"}\NormalTok{, }\DataTypeTok{header =}\NormalTok{ T)}
\KeywordTok{dim}\NormalTok{(house_data)}
\end{Highlighting}
\end{Shaded}

\begin{verbatim}
## [1] 21613    21
\end{verbatim}

\begin{Shaded}
\begin{Highlighting}[]
\CommentTok{# house_data$id <- NULL  # delete ID}
\CommentTok{# house_data$date <- NULL  # delete date}
\CommentTok{# separate training and testing data}
\KeywordTok{set.seed}\NormalTok{(}\DecValTok{2018}\NormalTok{)}
\NormalTok{indx <-}\StringTok{ }\KeywordTok{sample}\NormalTok{(}\KeywordTok{nrow}\NormalTok{(house_data), }\DataTypeTok{size =} \FloatTok{0.7}\OperatorTok{*}\KeywordTok{nrow}\NormalTok{(house_data))}
\NormalTok{house_train <-}\StringTok{ }\NormalTok{house_data[indx, ]  }\CommentTok{# training data}
\KeywordTok{dim}\NormalTok{(house_train)}
\end{Highlighting}
\end{Shaded}

\begin{verbatim}
## [1] 15129    21
\end{verbatim}

\begin{Shaded}
\begin{Highlighting}[]
\NormalTok{house_test <-}\StringTok{ }\NormalTok{house_data[}\OperatorTok{-}\NormalTok{indx, ]  }\CommentTok{# testing data}
\KeywordTok{dim}\NormalTok{(house_test)}
\end{Highlighting}
\end{Shaded}

\begin{verbatim}
## [1] 6484   21
\end{verbatim}


\end{document}
